\chapter{Einleitung}
\label{sec:Einleitung}

Diese Arbeit führt systematisch in die grundlegenden Konzepte und Terminologie des Fuzzings ein. Zunächst wird eine präzise Definition des Begriffs „Fuzzing“ gegeben und dessen Zielsetzung erläutert: die automatisierte Erzeugung und Ausführung zahlreicher, bewusst fehlerhafter oder ungewöhnlicher Eingaben mit dem Zweck, Robustheits-, Verfügbarkeits- und Sicherheitslücken in Software- und Systemkomponenten aufzudecken. Darauf aufbauend wird eine strukturierte Taxonomie von Fuzzing-Ansätzen vorgestellt — unter anderem die Unterscheidung in Black-Box, Gray-Box und White-Box Verfahren sowie die Differenzierung nach generativen und mutativen Techniken und nach Coverage-Guided vs. nicht-coverage-basierten, sowie nach Feedback und nicht Feedback orientierten Methoden. Weiterhin werden die maßgeblichen Klassifikationskriterien beschrieben, die Vergleiche und Evaluierungen von Fuzzern ermöglichen (u.\,a. Testfallgenerierung, Orakel-Strategien, Instrumentierungsgrad, Automatisierungsgrad und Skalierbarkeit). Im Anschluss werden typische Implementierungen dieser Algorithmen, wie AFL, vorgestellt. Abschließend werden die Grundlagen von HTTPS und TLS wiederholt, um einen methodisch sauberen Übergang zum Hauptthema zu schaffen.

Im Kern der Arbeit werden um die Einordnung der vorgestellten Konzepte und eine Aktualisierung und Aufarbeitung von aktuellen Forschungsdaten und Szenarien in der Praxis gehen.  
