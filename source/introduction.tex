\chapter{Einleitung}
\label{sec:Einleitung}

Diese Arbeit führt systematisch in die grundlegenden Konzepte und Terminologie des Netzwerkfuzzings ein. Zunächst werden die wichtigsten Begriffe und Definitionen erläutert, um ein gemeinsames Verständnis zu schaffen. Anschließend werden verschiedene Ansätze und Techniken des Netzwerkfuzzings vorgestellt und deren Vor- und Nachteile diskutiert. Dabei werden sowohl klassische, als auch intelligente Fuzzing-Methoden betrachtet. Abschließend wird ein Überblick über Tools und Frameworks gegeben, die im Bereich des Netzwerkfuzzings eingesetzt werden können. Dies ermöglicht dem Leser eine eigene Anwendung auf dem lokalen Gerät oder in einer Testumgebung. 

Das Ziel dieser Arbeit ist es, dem Leser ein fundiertes Wissen über Netzwerkfuzzing zu vermitteln, um die Sicherheit von Netzwerkanwendungen und -protokollen effektiv testen und verbessern zu können. Dafür werden aktuelle Forschungsergebnisse forgestellt.