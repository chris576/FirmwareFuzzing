
\chapter{Bonusmaterial}
\label{sec:Anhang}

Inhalte, die nicht im direkten Fokus der Aufgabenstellung stehen, jedoch zur Ausarbeitung indirekt beigetragen haben oder zum besseren Verständnis der dargestellten Aussagen beitragen, finden im Anhang der Arbeit eine passende Position.

\section{Messdaten}

Im begrenzten Umfang ist es hilfreich, im Anhang weiteres Datenmaterial der Arbeit hinzuzufügen, beispielsweise Tabellen, Messreihen oder kleinere Skripte, siehe Abschnitt~\ref{sec:Quellcode}.

Bei größeren Mengen an Daten oder Quellcode ist es jedoch sinnvoller, diese in elektronischer Form als Datei im Originalformat beizulegen, beispielsweise auf CD-ROM, USB-Stick, Multimedia Card (MMC) oder Down\-load im Repository einer Versionsverwaltung wie SVN oder Git.


\section{Formalien}

Der Inhalt ist wichtiger als die Verpackung. Dieser Grundsatz gilt insbesondere für eine Projekt- oder Abschlussarbeit. Dennoch gilt es einen gewissen Standard bei der Gestaltung der Ausarbeitung einzuhalten. Dieses Dokument kann beim Aufbau und der Gestaltung als Vorlage dienen. Um eine ingenieurwissenschaftliche Arbeit zu verfassen stehen die Standard Office-Produkte wie beispielsweise MS Word zur Verfügung. Word wurde jedoch nicht zum Verfassen wissenschaftlicher Arbeiten mit Formeln, Abbildungen und Referenzen konzipiert und dies macht sich im Laufe der Arbeit durch offensichtliche Unzulänglichkeiten schnell bemerkbar, wie beispielsweise die unterschiedliche Darstellung eines Word-Dokuments auf verschiedenen Rechnern mit abweichenden Word-Versionen. Weiterhin bedarf es sehr viel Aufwand und Zeit, bis ein Dokument annähernd so professionell gestaltet ist wie beispielsweise mit dem Textsetzprogramm \LaTeX, das zum Verfassen wissenschaftlicher Texte\footnote{\url{https://de.wikipedia.org/wiki/LaTeX}} geschaffen wurde. 

\section{Häufige Fehler}

Grobe Rechtschreibfehler sind durch eine oftmals verwendete Autokorrektur seltener geworden - im Bereich Zeichensetzung weisen Studierende jedoch erfahrungsgemäß oftmals Wissenslücken auf. 
Wenn Fragen zur Grammatik und Rechtschreibung bestehen, so sollte nicht gezögert werden, den Blick in ein Fachbuch zu werfen. Es muss nicht immer der Duden sein, es existieren auch flüssig geschriebene, kompakte Nachschlagewerke auf dem Markt, beispielsweise von Balcik und Röhe~\cite{balcik2010}. Die hierbei investierte Zeit wird sich im Laufe des Berufslebens schnell amortisieren!

Ebenso verhält es sich mit mathematischen Definitionen, die Klarheit schaffen können. Auch hier gilt: Für Projekt- oder Abschlussarbeit sollte das Wissen in diesem Bereich aufgefrischt werden - beispielsweise mit Fachbüchern aus den ersten Semestern~\cite{teschl2013}.

Noch ein letzter praktischer Hinweis: Das Inhaltsverzeichnis ist nicht Bestandteil des Inhaltsverzeichnisses!
